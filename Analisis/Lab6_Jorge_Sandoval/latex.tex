\documentclass[spanish,notitlepage,letterpaper, 12pt]{article} % para articulo en castellano
\usepackage{cite}
\usepackage[utf8]{inputenc}
\usepackage[ansinew]{inputenc} % Acepta caracteres en castellano
\usepackage[spanish]{babel} % silabea palabras castellanas
\usepackage{amsmath}

\usepackage{amsfonts}
\usepackage{amssymb}
\usepackage{hyperref} % navega por el doc
\usepackage{graphicx}
\usepackage{geometry}      % See geometry.pdf to learn the layout options.
\geometry{letterpaper}                   % ... or a4paper or a5paper or ... 
%\geometry{landscape}                % Activate for for rotated page geometry
%\usepackage[parfill]{parskip}    % Activate to begin paragraphs with an empty line rather than an indent
\usepackage{epstopdf}
\usepackage{fancyhdr} % encabezados y pies de pg

\usepackage{listings}
\usepackage{color}

\definecolor{dkgreen}{rgb}{0,0.6,0}
\definecolor{gray}{rgb}{0.5,0.5,0.5}
\definecolor{mauve}{rgb}{0.58,0,0.82}

\lstset{frame=shadowbox,
  language=Java,
  aboveskip=3mm,
  belowskip=3mm,
  showstringspaces=false,
  columns=flexible,
  basicstyle={\small\ttfamily},
  numbers=left,
  numberstyle=\tiny\color{gray},
  keywordstyle=\color{blue},
  commentstyle=\color{dkgreen},
  stringstyle=\color{mauve},
  breaklines=true,
  breakatwhitespace=true
  tabsize=3
  rulesepcolor=\color{blue}
}

\newcommand{\university}{\normalsize Universidad Industrial de Santander}
\newcommand{\faculty}{\normalsize  Escuela de Ingenier\'ia de Sistemas e Inform\'atica}
\newcommand{\codigo}{\normalsize  2182028}
\newcommand{\grupo}{\normalsize  B2}
\newcommand{\estudiante}{\normalsize  Jorge Sandoval}
\pagestyle{fancy} 
\chead{\bfseries Lab 6 } 
\lhead{} % si se omite coloca el nombre de la seccion
\rhead{\today} 
\lfoot{\it  An\'alisis N\'umerico } 
\cfoot{\university} 
\rfoot{\thepage} 

\voffset = -0.25in 
\textheight = 8.0in 
\textwidth = 6.5in
\oddsidemargin = 0.in
\headheight = 20pt 
\headwidth = 6.5in
\renewcommand{\headrulewidth}{0.5pt}
\renewcommand{\footrulewidth}{0,5pt}
\DeclareGraphicsRule{.tif}{png}{.png}{`convert #1 `dirname #1`/`basename #1 .tif`.png}


\begin{document}

\title{	\vspace{-12mm}\includegraphics[width=0.2\linewidth]{Logos/UIS.pdf}\\Informe Laboratorio: An\'alisis Num\'erico\\  \centering Pr\'actica No. 6}
\author{
\textbf{Estudiante:} \estudiante\\ \textbf{C\'odigo:} \codigo\\
\textbf{Grupo:} \grupo\\
\textit{\faculty}\\
\textit{\university}}
\date{\today}
\maketitle

\section*{3.2 Applying}

\subsection*{A.} Sabiendo que $Y(x)=\sqrt{x}$, usando $x_0=1$, $x_1=1.25$, $x_2=1.5$ y $y_1=1$, $y_2=\sqrt{1.25}=1.1180$, $y_3=1.2247$. Tenemos que 

\begin{equation}
    P_2=1\frac{(x-1.25)(x-1.5)}{(1-1.25)(1-1.5)}+1.1180\frac{(x-1)(x-1.5)}{(1.25-1)(1.25-1.5)}+1.2247\frac{(x-1)(x-1.25)}{(1.5-1)(1.5-1.25)}
\end{equation}
Operando y factorizando la ecuacion (1) tenemos

\begin{equation}
    8.08321x^2-22.51915x+15.3
\end{equation}
\subsection*{B.}

\begin{center}
\begin{tabular}{|c|c|c|c|c|c|} 
\hline
x_k & f[x_k] & f[,]& f[,,]& f[,,,] \\
\hline
x_0=1.0 & 3.5 & - & - & - \\ 
x_0=1.5 & 12 & 17 & - & - \\ 
x_0=3.5 & 103 & 45.5 & 11.4 & - \\ 
x_0=5.0 & 491.5 & 259 & 61 & 12.4 \\ 
\hline
\end{tabular}
\end{center}
$P_2(x)=3.5+17(x-1)+11.4(x-1)(x-1.5)+12.4(x-1)(x-1.5)(x-3.5)$
\section*{3.3 Implementing}
\subsection*{D.} 

De las gráficas se puede observar que son similares por ende se puede inferir que dichos métodos son parecidos y que sus margen de error es pequeño.
\end{document}  




