\documentclass[spanish,notitlepage,letterpaper, 12pt]{article} % para articulo en castellano
\usepackage{cite}
\usepackage[utf8]{inputenc}
\usepackage[ansinew]{inputenc} % Acepta caracteres en castellano
\usepackage[spanish]{babel} % silabea palabras castellanas
\usepackage{amsmath}

\usepackage{amsfonts}
\usepackage{amssymb}
\usepackage{hyperref} % navega por el doc
\usepackage{graphicx}
\usepackage{geometry}      % See geometry.pdf to learn the layout options.
\geometry{letterpaper}                   % ... or a4paper or a5paper or ... 
%\geometry{landscape}                % Activate for for rotated page geometry
%\usepackage[parfill]{parskip}    % Activate to begin paragraphs with an empty line rather than an indent
\usepackage{epstopdf}
\usepackage{fancyhdr} % encabezados y pies de pg

\usepackage{listings}
\usepackage{color}

\definecolor{dkgreen}{rgb}{0,0.6,0}
\definecolor{gray}{rgb}{0.5,0.5,0.5}
\definecolor{mauve}{rgb}{0.58,0,0.82}

\lstset{frame=shadowbox,
  language=Java,
  aboveskip=3mm,
  belowskip=3mm,
  showstringspaces=false,
  columns=flexible,
  basicstyle={\small\ttfamily},
  numbers=left,
  numberstyle=\tiny\color{gray},
  keywordstyle=\color{blue},
  commentstyle=\color{dkgreen},
  stringstyle=\color{mauve},
  breaklines=true,
  breakatwhitespace=true
  tabsize=3
  rulesepcolor=\color{blue}
}

\newcommand{\university}{\normalsize Universidad Industrial de Santander}
\newcommand{\faculty}{\normalsize  Escuela de Ingenier\'ia de Sistemas e Inform\'atica}
\newcommand{\codigo}{\normalsize  2182028}
\newcommand{\grupo}{\normalsize  B2}
\newcommand{\estudiante}{\normalsize  Jorge Sandoval}
\pagestyle{fancy} 
\chead{\bfseries Lab 9 } 
\lhead{} % si se omite coloca el nombre de la seccion
\rhead{\today} 
\lfoot{\it  An\'alisis N\'umerico } 
\cfoot{\university} 
\rfoot{\thepage} 

\voffset = -0.25in 
\textheight = 8.0in 
\textwidth = 6.5in
\oddsidemargin = 0.in
\headheight = 20pt 
\headwidth = 6.5in
\renewcommand{\headrulewidth}{0.5pt}
\renewcommand{\footrulewidth}{0,5pt}
\DeclareGraphicsRule{.tif}{png}{.png}{`convert #1 `dirname #1`/`basename #1 .tif`.png}


\begin{document}

\title{	\vspace{-12mm}\includegraphics[width=0.2\linewidth]{Logos/UIS.pdf}\\Informe Laboratorio: An\'alisis Num\'erico\\  \centering Pr\'actica No. 9}
\author{
\textbf{Estudiante:} \estudiante\\ \textbf{C\'odigo:} \codigo\\
\textbf{Grupo:} \grupo\\
\textit{\faculty}\\
\textit{\university}}
\date{\today}
\maketitle

\section*{3.1 Applying}


\subsection*{A.}

$h=\dfrac{1}{4}$ \hspace{1cm} $\dfrac{b-a}{N-1}=\dfrac{1}{4}$ despejando para N tenemos que $N=5$.
\begin{equation*}
    \int_1^2 xln(x) dx=\int_1^1.25 xln(x)dx + \int_1.25^1.5 xln(x)dx + \int_1.5^1.75 xln(x)dx + \int_1.75^2 xln(x)dx
\end{equation*}
\begin{equation*}
    \dfrac{h}{2}(f_0+f_1)+\dfrac{h}{2}(f_1+f_2)+\dfrac{h}{2}(f_2+f_3)+\dfrac{h}{2}(f_3+f_4)+\dfrac{h}{2}(f_4+f_5)
\end{equation*}
\begin{equation*}
    =\dfrac{h}{2}(f_0+2f_1+2f_2+2f_3+f4)
\end{equation*}
\begin{equation*}
    \frac{1}{8}(ln(1)+2(1.25)ln(1.25)+2(1.5)ln(1.5)+2(1.75)ln(1.75)+ln(2))=0.5532570
\end{equation*}


\subsection*{B.}
tenemos que N=5 así que
\begin{equation*}
\frac{h}{3}(f_0+4f_1+2f_2+4f_3+f_4)=\dfrac{1}{12}(ln(1)+4(1.25)ln(1.25)+2(1.5)ln(1.5)+4(1.75)ln(1.75)+ln(2))
\end{equation*}
\begin{equation*}
    =0.5784476
\end{equation*}
\section*{3.4 Proposing}
\subsection{a.} Hallar el área limitada por la recta x + y = 10, el eje OX y las ordenadas de x=2 y x=8.

\begin{figure}[h]
    \centering
    \includegraphics{Images/Captura de pantalla_2021-06-22_14-20-33.png}
    \caption{grafica}
\end{figure}

\end{document}  



