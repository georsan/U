\documentclass[spanish,notitlepage,letterpaper, 12pt]{article} % para articulo en castellano
\usepackage{cite}
\usepackage[utf8]{inputenc}
\usepackage[ansinew]{inputenc} % Acepta caracteres en castellano
\usepackage[spanish]{babel} % silabea palabras castellanas
\usepackage{amsmath}

\usepackage{amsfonts}
\usepackage{amssymb}
\usepackage{hyperref} % navega por el doc
\usepackage{graphicx}
\usepackage{geometry}      % See geometry.pdf to learn the layout options.
\geometry{letterpaper}                   % ... or a4paper or a5paper or ... 
%\geometry{landscape}                % Activate for for rotated page geometry
%\usepackage[parfill]{parskip}    % Activate to begin paragraphs with an empty line rather than an indent
\usepackage{epstopdf}
\usepackage{fancyhdr} % encabezados y pies de pg

\usepackage{listings}
\usepackage{color}

\definecolor{dkgreen}{rgb}{0,0.6,0}
\definecolor{gray}{rgb}{0.5,0.5,0.5}
\definecolor{mauve}{rgb}{0.58,0,0.82}

\lstset{frame=shadowbox,
  language=Java,
  aboveskip=3mm,
  belowskip=3mm,
  showstringspaces=false,
  columns=flexible,
  basicstyle={\small\ttfamily},
  numbers=left,
  numberstyle=\tiny\color{gray},
  keywordstyle=\color{blue},
  commentstyle=\color{dkgreen},
  stringstyle=\color{mauve},
  breaklines=true,
  breakatwhitespace=true
  tabsize=3
  rulesepcolor=\color{blue}
}

\newcommand{\university}{\normalsize Universidad Industrial de Santander}
\newcommand{\faculty}{\normalsize  Escuela de Ingenier\'ia de Sistemas e Inform\'atica}
\newcommand{\codigo}{\normalsize  2182028}
\newcommand{\grupo}{\normalsize  B2}
\newcommand{\estudiante}{\normalsize  Jorge Sandoval}
\pagestyle{fancy} 
\chead{\bfseries Lab 6 } 
\lhead{} % si se omite coloca el nombre de la seccion
\rhead{\today} 
\lfoot{\it  An\'alisis N\'umerico } 
\cfoot{\university} 
\rfoot{\thepage} 

\voffset = -0.25in 
\textheight = 8.0in 
\textwidth = 6.5in
\oddsidemargin = 0.in
\headheight = 20pt 
\headwidth = 6.5in
\renewcommand{\headrulewidth}{0.5pt}
\renewcommand{\footrulewidth}{0,5pt}
\DeclareGraphicsRule{.tif}{png}{.png}{`convert #1 `dirname #1`/`basename #1 .tif`.png}


\begin{document}

\title{	\vspace{-12mm}\includegraphics[width=0.2\linewidth]{Logos/UIS.pdf}\\Informe Laboratorio: An\'alisis Num\'erico\\  \centering Pr\'actica No. 8}
\author{
\textbf{Estudiante:} \estudiante\\ \textbf{C\'odigo:} \codigo\\
\textbf{Grupo:} \grupo\\
\textit{\faculty}\\
\textit{\university}}
\date{\today}
\maketitle

\section*{3.1 Applying}

\subsection*{A.}
\begin{equation*}
F^(x)=\frac{2f(x-h)-f(x)+f{(')}(x)h}{h^2}=\frac{2f(x-h)-f(x)+2(\frac{f(x+h)-f(x-h)}{2h})h}{h^2}= 
\end{equation*}
\begin{equation*}
\frac{f(x-h)-f(x)+f(x+h)}{h^2}=\frac{cos(1-0.05)-cos(1)+cos(1+0.05)}{0.05^2}=-0.540189752
\end{equation*}
\subsection*{B.}
\begin{equation*}
F^(x)=\frac{2f(x-h)-f(x)+f{(')}(x)h}{h^2}=\frac{2f(x-h)-f(x)+2(\frac{f(x+h)-f(x-h)}{2h})h}{h^2}= 
\end{equation*}
\begin{equation*}
\frac{f(x-h)-f(x)+f(x+h)}{h^2}=\frac{cos(1-0.01)-cos(1)+cos(1+0.01)}{0.01^2}=-0.54029780
\end{equation*}

\subsection*{C.}
\begin{equation*}
    f(x+h)-f(x-h)=2f^{'}+\frac{f^{(3)}h^3}{3!}=f(x+h)-f(x-h)=2f^{'}+2(f(x)-f^{'}(x)+\frac{f^{(2)(x)h^2}}{2!}-f(x-h))
\end{equation*}
\begin{equation*}
    f^{''}(x)=\frac{f(x+h)+f(x-h)-2f(x)}{h^2}=\frac{cos(1+0.1)+cos(1-0.1)-2cos(1)}{0.1^2}=-0.539852204
\end{equation*}

\subsection*{D.}
La respuesta mas precisa es la B.
\subsection*{3.1 B}
 \begin{equation*}
    f{''}(x_0)=\frac{2f_0-5f_1+4f_2-f_3}{h^2}=\frac{2(0.629492)-5(0.610192)+4(0.592710)-(0.577125)}{0.05^2}=0.6956
 \end{equation*}

\end{document}  



